

\documentclass{article}
\usepackage{times}


\begin{document}

\title{Agar.io}
\author{Yash Gopal, Vicky Bilbily, and Jemar Jones.\\
  SE 3XA3\\}
\date{}
\maketitle


\section*{Intro}
Agar.io is a simple online multiplayer game. Essentially, each player is a circle, that can move, that gets bigger by eating other circles. The goal of the game is to get the highest score by eating the most circles. A player loses when they are eaten by another circle. 
\section*{What is problem we're trying to solve?}
Most people love casual games: whether it's on mobile or on the web. The problem with most of these is that they usually try to do too much and get overly complicated. There is also a huge lack of simple, arcade style multiplayer games that used to keep us hooked to the screens like in the early days of the internet. We just want to make a simple, free, fun yet challenging game that anyone can play with hundreds of other people or friends as long as they have access to the internet. 
We decided to develop this game because there are many challenging components to it. The gameplay mechanics are seemingly simple, but have quite a lot of little constraints and niche scenarios that need to be accounted for.  By developing Agar.io, we are trying to learn about full-stack web technologies, especially working with servers. 
\section*{Why is this important?}
This problem is important because there is serious lack of simple, fun, mobile/web based multiplayer games which are free in the market. There is a huge demand for games like that. Most multiplayer games nowadays also require you to register or login which is also a huge deterrent to most people. In this game all you need is a custom name and you can join a server to start playing. This also means you can play with your friends, wherever they are and whenever you want.  
On an another note making open source projects and letting other people reiterate on them and make them better is one of the core principles of the software development community and the tech industry. We strongly embrace this concept. As an open source game it should therefore provide inspiration for other creators and developers to make simple, multiplayer games. 
\section*{What is the context and scope of this problem? Who are the stakeholders?} 
The scope of the problem  is to develop Agar.io in its entirety from the front-end to the back-end, with online multiplayer enabled.  
The stakeholders are anyone who has access to the URL where our game is hosted since the project is going to be deployed on a public server. The target demographic are primarily young people between the ages of 10 and 25 who have access to internet. 
\end{document}